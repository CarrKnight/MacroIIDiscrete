\documentclass[a4paper,12pt,english]{beamer}
\usepackage{graphicx}
% Additional Options
\usepackage{natbib}
\usepackage{caption}
\usepackage{subcaption}
\usepackage{amsmath}
\usepackage{float}

\AtBeginSection[]
{
 \begin{frame}<beamer>
 \frametitle{Plan}
 \tableofcontents[currentsection]
 \end{frame}
}

% Additional Options
\begin{document}
\title{Summer Summary}
\author{Ernesto Carrella}


\begin{frame}
\titlepage
\end{frame}

\section{Introduction}

\begin{frame}
	\frametitle{Summer wrapup}
	\begin{enumerate}
		\item  Waiting for JASSS review
		\pause
		\item Learning Marginal Costs/Benefits
		\pause
		\item Inventory Control + Price Determination				
	\end{enumerate}
	
\end{frame}


\section{JASSS Review}

\begin{frame}
\frametitle{JASSS Article}
\begin{itemize}
\item I submitted the old PID article to JASSS on June 5th
\pause
\item Haven't heard back yet
\pause
\item When I do, I might have to stop what else I am doing to revise and resubmit.
\end{itemize}
\end{frame}

\section{Learning Marginal Costs/Benefits}

\begin{frame}
\frametitle{Why Learning?}
\begin{itemize}
\item Firms used to use only an hill-climber algorithm to choose how much to produce
\pause
\item Hill-climber is too limited, doesn't work with supply chains.
\pause
\item Switched to periodically checking marginal benefits versus marginal costs of hiring a new worker.
\pause
\item Current prices and wages are a poor indicator of the real marginal benefit and marginal costs, so the firm needs to learn them separately
\end{itemize}
\end{frame}

\begin{frame}
\frametitle{Learning Solution so far}
\pause
\begin{itemize}
\item The tricky part is that in classical economics knowing marginal costs/benefits is what separates monopolist from competitive. 
\pause
\begin{itemize}
\item The profit of a competitive firm is:
\[ \Pi = p q - w l \]
\item The profit of a monopolist firm is:
\[ \Pi = p(q) q - w(l) l \]
\end{itemize}
The competitor takes prices as given, the monopolist as dependent on its own production.
\pause
\item I didn't want to code two separate behaviors, so instead I wanted a learning behavior that would predict fixed prices for competitive firms and variable prices for the monopolist
\pause
\item Each firm a (simple) econometrician.
\end{itemize}
\end{frame}


\section{Inventory Control}

\begin{frame}
\frametitle{Inventory Control + Price Determination}
\begin{itemize}
\item Currently firms use PID to set prices, using the difference
\[ \text{ Items on Sale } - \text{ Number of Customers} \]
as feedback
\pause
\item What if the number of customers is not visible? PID still works using
\[ \text{ Today's Inventory } - \text{ Yesterday's inventory} \]
\pause
\item Focus only on $\Delta$ of the inventory, not its stock
\pause
\item What if the firm wants to target a specific level of inventory while at the same time find the right price?
\end{itemize}
\end{frame}

\begin{frame}
\frametitle{Best solution so far}
\pause
\begin{itemize}
\item It is possible to achieve the two targets at the same time.
\pause
\item The idea is that the PID selecting price uses as feedback:
\[ \text{ Today's Inventory } - \text{ Yesterday's inventory} + \text{ Adjustment} \]
\pause
\item The adjustment is 0 when inventory is at the right level, otherwise it's a proportion of how distant we are from target inventory.
\pause
\item This way the PID automatically finds a price that generates an inflow when we need to increase our inventory and viceversa.
\end{itemize}
\end{frame}

\begin{frame}
\frametitle{The problem with inventory}
\begin{itemize}
\item Inventory work really well when there is only one sector
\pause
\item As soon as we move to supply chains, inventories just cause over-production and then crashes
\end{itemize}
\end{frame}

\section{This Week}

\begin{frame}
\frametitle{this week plan}
\begin{itemize}
\item Literature review: inventories and cycles
\pause
\item Sensitivity analysis: does inventory size matter in supply-chains fluctuations?
\end{itemize}
\end{frame}





\end{document}