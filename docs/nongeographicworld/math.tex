\documentclass[]{article}


\usepackage{amsmath}
\usepackage{amsfonts}
\usepackage{amssymb}
%opening
\title{Simple Macro}
\author{Ernesto}
\begin{document}

\maketitle

\section{Simple world}
There are $N$ people in the world. Each day a person has to decide whether to be farmer or worker.  
As worker, a person gets daily wage $w$ and produces a fixed number $\gamma$ of manufactured goods. 
As farmer, the person produces $\eta$ units of agricultural goods every day.
$\eta$ is heterogeneous: the first farmer was born with $\eta=1$, the second with $\eta=2$, and the last $\eta = N$. 
Personal agricultural production $\eta$ is not a function of how many other people farm, it is a personal parameter of each agent.

In aggregate, $L_m$ is the number of people who works in manufacturing, $L_a$ in farming. 
\[ L_a + L_m = N \]
Given $L_m$ workers, total daily production is:
\[ Q_m = \gamma L_m \]
A farmer will become a worker only if wages are above or equal its personal $\eta$. 
Because I expect the "worst" farmers (those with the lowest $\eta$) to become workers first, daily wages are:
\[ w_m \geq L_m\]
The $L_a$ people with the highest $\eta$s will be farmers, total agricultural production is:
\[ q_a =  \frac{N(N+1)}{2} - \frac{L_m(L_m+1)}{2} \]
(that's the sum of all integers between $L_m+1$ and $N$ )which simplifies in:
\[q_a = \frac{L_a(N+1) -L_aL_m}{2} \]

Every person has the same utility function:
\[ U = (q_a+1)^{.5}(q_m+1)^{.5} \]
setting MRS equal to price ratio, and remembering that $a$ is the monetary commodity so its price is one:
\begin{align*}
\text{MRS} &= \frac{p_m}{p_a} \\
\frac{\frac{\partial U}{\partial m}}{\frac{\partial U}{\partial a}} &= p_m \\
q_a &= p_m(q_m+1)-1
\end{align*}
For workers, the budget constraint is:
\[ q_a + p_m q_m = L_m  \]
Where $L_m$ is the wage.
\[  p_m(q_m+1)-1 + p_m q_m = L_m  \]
\[ q_m = \frac{L_m-p_m+1}{2 p_m}  \]
A farmer budget constraint is contingent on its own personal production:
\[q_a + p_m q_m = \eta_i \]
\[ p_m(q_m+1)-1 + p_m q_m = \eta_i \]
\[ q_m = \frac{\eta_i-p_m+1}{2 p_m} \]

Let's aggregate demands. We sum up the demand for all workers, which are all identical and the demand for farmers which are heterogeneous:
\[Q_m = L_m \times \text{Worker demand}) + \sum \text{Farmer Demand}\]
\[ Q_m = L_m \left( \frac{L_m}{2p_m} + \frac{1 - p_m}{2p_m} \right) +  \sum \left( \frac{\eta}{2p_m} + \frac{1 - p_m}{2p_m} \right) \]
Now, $\sum \eta_i$ is just total agricultural production $Q_a$.
\[ Q_m = \frac{L^2_m}{2p_m} + L_m\frac{1 - p_m}{2p_m}  +   \frac{Q_a}{2p_m} + L_a\frac{1 - p_m}{2p_m}  \]
$L_a + L_m$ is everyone so:
\[ Q_m = \frac{L^2_m}{2p_m} +   \frac{Q_a}{2p_m} + N\frac{1 - p_m}{2p_m}  \]
\[ Q_m = \frac{L^2_m + Q_a +N(1 - p_m)}{2p_m} \]
Or if we want the price:
\[ p_m = \frac{q_a + L_m^2 + N}{2 q_m + N}\]

Notice that $L_m$ is really a function of $q$ so we could simply further, but for now keep it like that.

\section{Zero profits}
Manufacturing must make zero profits
\[ p_m q_m - w L_m = 0\]
\[ p_m \frac{L^2_m + Q_a +N(1 - p_m)}{2p_m} - L^2_m = 0\]
Now if we add this condition, we have a sistem of 5 equations in 5 unknowns (after setting $N=50$, $\gamma=10$):
\[
\begin{cases}  p_m \frac{L^2_m + Q_a +50(1 - p_m)}{2p_m} - L^2_m = 0 & \text{Profits are 0 }
\\ Q_a =  \frac{50(50+1)}{2} - \frac{L_m(L_m+1)}{2} & \text{Agricultural production definition } \\
 L_a + L_m = N &\text{Everybody is employed } \\ 
Q_m = 10 L_m &\text{Linear manufacturing production } \\
p_m = \frac{Q_a + L_m^2 + 50}{2 Q_m + 50} &\text{Demand function }
 \end{cases} 
\]

And you end up with
\begin{align*}
L_m &= 27.9441 \approx 28 \\
L_a &\approx 22 \\
Q_m &\approx 280 \\
Q_a &\approx 869 \\
p &\approx 2.70 \\
\end{align*}


If $N=200$ instead:
\begin{align*}
L_m &= 109.7 \approx 110 \\
L_a &\approx 90 \\
Q_m &\approx 1100 \\
Q_a &\approx 13995 \\
p &\approx 10.95 \\
\end{align*}
\end{document}
